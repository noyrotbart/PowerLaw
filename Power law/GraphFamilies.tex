% !TEX root = WWW.tex

 \section{Power-Law Graphs}\label{Sec:GraphFamilies}

In the literature \emph{power-law} graphs are usually defined as the class of $n$ vertex graphs $G$ such that $\mathrm{ddist}_G(k)$
is proportional to $k^{-\alpha}$ for some real number $\alpha > 1$. Ideally, and ignoring rounding,  $\mathrm{ddist}_G(k) = Ck^{-\alpha}$ for all $k$ for constant $C$. As the degree distribution of a graph must be a probability distribution, we have $\sum_{k=1}^\infty C k^{-\alpha} = C \sum_{k=1}^\infty k^{-\alpha} = 1$, hence  $C = 1/\zeta(\alpha)$ where $\zeta$ is the Riemann zeta function. However, in the literature, concessions are usually made that relax
the restrictions on $\mathrm{ddist}_G(k)$, for example that the power-law property need only hold for high-degree
vertices (``above a cutoff''), or that $\mathrm{ddist}_G(k)$ is only \emph{approximately} equal to $Ck^{-\alpha}$,
with some approximation error that falls off with $n$.
To ensure that our results hold for all these variations of power-law graphs, 
 we define two families of graphs $\PLB$ and $\PLC$ with $\PLC\subsetneq\PLB$. Family $\PLB$ is rich enough to contain the graphs whose degree distribution is approximately, or perfectly, power-law distributed, and our upper bound on the label size for our labeling scheme holds for any graph in $\PLB$. Family $\PLC$ is used to show our lower bound
 and is restrictive enough that most definitions of power-law graph occurring in the literature will contain it. 
 
In the following, let $i_1 = \Theta(\sqrt[\alpha]n)$ be the smallest integer such that $\lfloor Cn/i_1^\alpha\rfloor \leq 1$, and let $C'\geq(\frac C{\alpha-1} + \frac{i_1}{\sqrt[\alpha] n} + 5)^{\alpha} + \frac{C}{\alpha - 1}$ be a constant; we shall use $C'$ in the remainder of the paper.
\begin{definition} \label{def:general-family}
Let $\alpha > 1$ be a real number and let $\chi:\mathbb N\rightarrow\mathbb N$ be a function. $\PLBARG{\chi}{\alpha}$ is the family of graphs $G$ such that if $n = \vert V(G)\vert$ then for all integers $k$ between $\chi(n)$ and $n-1$, $\sum_{i = k}^{n-1} {\vert V_i\vert} \leq C'(\frac{n}{k^{\alpha-1}})$. We shall usually suppress $\chi$ and $\alpha$, writing merely $\PLB$.
\end{definition}
The function $\chi$ captures the notion of a cutoff as defined in~\cite{clauset2009power} (Sec. 3.1); the intuition is  that for an $n$-vertex graph the power-law distribution need only apply for nodes of degree higher than $\chi(n)$, rather than for all degrees. 
Setting $\chi(n) = 1$ corresponds to the case where the entire range of degrees follows a power-law distribution, hence
even for small values of $\chi(n)$, $\PLB$ morally contains all graphs with power-law degree distribution.
We will later prove upper bounds that hold for \emph{all} $\chi$ bounded from above by some function; in particular
for the upper bound for adjacency labelling schemes, the bound holds for $\chi(n)$ as high as  $\sqrt[\alpha]{n/\log n}$.

The class $\PLC$ contains graphs where  the number of vertices of degree $k$
must be $C \frac{n}{k^{\alpha}}$ rounded either up or down and the number of vertices of degree $k$ is non-increasing
with $k$. Note that the function $k \mapsto  C \frac{1}{k^{\alpha}}$ is strictly decreasing.

\begin{definition}\label{def:proper}
Let $\alpha > 1$ be a real number and let $C = 1/\zeta(\alpha)$ where $\zeta$ is the Riemann zeta function. $\PLCARG{\alpha}$ is the set of graphs  $G=(V,E)$ such that
\begin{enumerate}
\item $\lfloor Cn\rfloor - i_1 - 1\leq\vert V_1\vert\leq\lceil Cn\rceil$,
\item $\lfloor C\frac n{2^\alpha}\rfloor\leq\vert V_2\vert\leq\lceil C\frac n{2^\alpha}\rceil + 1$,
\item for every $i$ with $3 \leq i \leq n$:  $\vert V_i \vert\in \{\lfloor C\frac{n}{i^{\alpha}} \rfloor, \lceil C \frac{n}{i^{\alpha}} \rceil\}$, and
\item for every $i$ with $2 \leq i \leq n-1$: $\vert V_i \vert \geq \vert V_{i+1} \vert$.
\end{enumerate}
We usually suppress $\alpha$, writing just $\PLC$.
\end{definition}
Note that we allow slightly more noise in the sizes of $V_1$ and $V_2$ than in the remaining sets; without it, it seems tricky to prove a better lower bound than $\Omega(\sqrt[\alpha+1]{n})$ on label sizes.

We show the following properties of $\PLC$. 
%The proofs of Propositions~\ref{prop:powerlawsparse} and~\ref{prop:Contained} are in App.~\ref{App:GraphFamilies}.
\begin{proposition}\label{prop:maxvertexproper}\label{prop:maxdegree}
The maximum degree in an $n$-vertex graph in $\PLC$ is at most $\left(\frac{C}{\alpha - 1} + 2\right) \sqrt[\alpha]{n} + i_1 + 3 = \Theta(\sqrt[\alpha]n)$.
\end{proposition}

\begin{proof}
Let $n > 0$ be an integer and let $k' = \lfloor \sqrt[\alpha]{n} \rfloor$. 
Furthermore, let $S_{k'} = \sum_{i=1}^{k'} \vert V_i\vert$, that is $S_{k'}$ is the number of vertices of degree at most $k'$. Let $S^{-}_{k'} = (\sum_{i=1}^{k'} \lfloor Cni^{-\alpha}\rfloor) - i_1 - 1$. Then
$S_{k'} \geq S^{-}_{k'}$. We now bound $S^{-}_{k'}$ from below.
For every $i$ with $1 \leq i \leq k'$,
\begin{align*}
S^{-}_{k'} + k' & = -i_1 - 1 + \sum_{i=1}^{k'} \left(\left\lfloor Cn i^{-\alpha}\right\rfloor + 1\right) \geq \\ 
 &-i_1 - 1 + \sum_{i=1}^{k'} Cn i^{- \alpha}  = -i_1 - 1 + Cn \sum_{i=1}^{k'} i^{-\alpha} \\
    & \geq n \left(1 - C\sum_{i=k'+1}^{\infty} i^{-\alpha} \right) - i_1 - 1 \\
    & \geq n \left( 1 - C\int_{k'}^\infty x^{-\alpha} dx \right) - i_1 - 1\\
 & = n \left( 1 - C\left[ \frac{1}{\alpha - 1} x^{-\alpha + 1}\right]_{\infty}^{k'}\right) - i_1 - 1 \\
 & = n \left( 1 - \frac{C}{\alpha - 1} \left( \lceil\sqrt[\alpha]{n} \rceil \right)^{-\alpha + 1}\right) - i_1 - 1\\
 & \geq n \left( 1 -  \frac{C}{\alpha - 1} \left(\sqrt[\alpha]{n}\right)^{-\alpha + 1} \right) - i_1 - 1 \\
 & = n - \frac{Cn}{\alpha - 1}n^{-1+\frac{1}{\alpha}} - i_1 - 1\\
 & = n - \frac{C}{\alpha - 1}\sqrt[\alpha]{n} - i_1 - 1,
\end{align*}
giving $S_{k'} \geq S^{-}_{k'} \geq n - \frac{C}{\alpha - 1}\sqrt[\alpha]{n} - \lceil \sqrt[\alpha]{n} \rceil - i_1 - 1$. There are thus at most $\frac{C}{\alpha - 1} \sqrt[\alpha]{n} + \lfloor \sqrt[\alpha]{n} \rfloor + i_1 + 1$ vertices of degree strictly more than $k' = \lceil \sqrt[\alpha]{n} \rceil$. Since for every $1 \leq i \leq n-1$: $\vert V_i \vert \geq \vert V_{i+1} \vert$, it follows that the maximum degree of any graph in $\PLC$ is at most $\left(\frac{C}{\alpha - 1} + 2\right) \sqrt[\alpha]{n} + i_1 + 3$.
\end{proof}

\begin{proposition}\label{prop:powerlawsparse}
For $\alpha > 2$, all graphs in $\PLC$ are sparse.
\end{proposition}
\begin{proof}
By Proposition \ref{prop:maxvertexproper}, the maximum degree of an $n$-vertex graph in $\PLC$
graph is at most $k' \triangleq \left(\frac{C}{\alpha - 1} + 2\right) \sqrt[\alpha]{n} + i_1 + 3$, whence
the total number of edges is at most $\frac{1}{2}\sum_{k=1}^{k'} k \vert V_k\vert$. By definition,
$\vert V_k \vert\leq \lceil \frac{Cn}{k^\alpha}\rceil \leq \frac{Cn}{k^{\alpha}} + 1$ for $k\neq 2$ and $\vert V_2\vert\leq\lceil\frac{Cn}{2^{\alpha}}\rceil + 1$, and thus
\begin{align*}
\frac{1}{2}\sum_{k=1}^{k'} k\vert V_k\vert \leq 1 + \frac{1}{2}\sum_{k=1}^{k'} k \left(\frac{Cn}{k^{\alpha}} + 1 \right) \\
\leq 1 + \frac{k'(k'+1)}{4} + Cn\sum_{k=1}^{\infty} k^{-\alpha+1} \\
= O(n^{2/\alpha}) + Cn \zeta(\alpha - 1) = O(n).
\end{align*}
\end{proof}

\begin{proposition}\label{prop:Contained}
For any $\chi$ and $\alpha > 1$,
$\PLCARG{\alpha} \subseteq \PLBARG{\chi}{\alpha}$.
\end{proposition}
\begin{proof}
Let $d = \lfloor(\frac C{\alpha - 1} + 2)\sqrt[\alpha]{n} + i_1 + 3\rfloor$. For any graph in $\PLC$ with $n$ vertices and for any $k$, $\vert V_k\vert\leq Ck^{-\alpha}n + 1$ and by Proposition~\ref{prop:maxvertexproper}, $\vert V_k\vert = 0$ when $k > d$.

Let $k$ be an arbitrary integer between $\chi(n)$ and $n-1$. We need to show that $\sum_{i = k}^{n-1} {\vert V_i\vert} \leq C'(\frac{n}{k^{\alpha-1}})$. It suffices to show this for $k\leq d$. We have:
\begin{align*}
  \sum_{i = k}^{n-1} {\vert V_i\vert} & \leq \sum_{i = k}^d(Ci^{-\alpha}n + 1)=    d - k + 1 + Cn\sum_{i = k}^d i^{-\alpha}\\
  & \leq \left(\frac C{\alpha - 1} + \frac{i_1}{\sqrt[\alpha]n} + 5\right)\sqrt[\alpha]{n} + Cn\int_k^d x^{-\alpha}dx\\
  & \leq \left(\frac C{\alpha - 1} + \frac{i_1}{\sqrt[\alpha]n} + 5\right)\sqrt[\alpha]{n} + Cn\left[\frac 1{\alpha - 1}x^{-\alpha+1}\right]_\infty^k\\
  & \leq \left(\left(\frac C{\alpha - 1} + \frac{i_1}{\sqrt[\alpha]n} + 5\right)\left(\frac{\sqrt[\alpha]nd^{\alpha-1}}n\right) + \frac {C}{\alpha - 1}\right)nk^{-\alpha +1}\\
  & \leq \left(\frac C{\alpha - 1} + \frac{i_1}{\sqrt[\alpha]n} + 5\right)\left(\frac C{\alpha-1} + \frac{i_1}{\sqrt[\alpha]n} + 5\right)^{\alpha - 1}  nk^{-\alpha+1} \\
  & +  \left(\frac{C}{\alpha - 1}\right) nk^{-\alpha+1} \\
  & \leq C'nk^{-\alpha+1},
\end{align*}
as desired.
\end{proof}

\subsection{Comparison to other deterministic models}
Numerous probabilistic and  deterministic  definitions of power-law graphs are given in the literature.
A recent deterministic model, called  shifted power-law distribution \cite{eom2011characterizing} has recently proven to capture a vast number of such definitions, both in theory and experimentally in \cite{Sankowski2016PowerLaw}.
We show that our definition of $\PLB$ contains graphs that adhere to the model, which is defined as follows.
 Let $c_1 > 0$ be a constant. A graph $G$ is \emph{power-law bounded} for parameters $\alpha > 1$ and $t\geq 0$ if for every integer $d\geq 0$, the number of vertices of $G$ of degree in $[2^d,2^{d+1})$ is at most
\[
  c_1n(t+1)^{\alpha - 1}\sum_{i = 2^d}^{2^{d+1} - 1}(i+t)^{-\alpha}.
\]
As experimentally verified in \cite{Sankowski2016PowerLaw}, the value of $t$ is typically very small. If $t = O(1)$, the bound above becomes $O(n\sum_{i = 2^d}^{2^{d+1} - 1}i^{-\alpha})$. In this case, our family $\PLB(\chi,\alpha)$ is rich enough to contain these power-law bounded graphs for sufficiently large $C'$ and any choice of $\chi$ and $\alpha$. This follows since for any power-law bounded graph with $n$ vertices and any integer $k$ between $1$ and $n-1$, $\sum_{i = k}^{n-1}{\vert V_i\vert} = O(\sum_{d = \lfloor\lg k\rfloor}^{\lceil\lg(n-1)\rceil}n\sum_{i = 2^d}^{2^{d+1}-1}i^{-\alpha}) = O(\frac{n}{k^{\alpha-1}})$. Thus our upper bound also applies to power-law bounded graphs. It is possible to extend our upper bound to super-constant $t$ where the bound is stronger the smaller $t$ is; we omit the details. Conversely, our family $\PLC$ is restrictive enough that
$\PLC$ is contained in the family of power-law bounded graphs when $t = O(1)$, and the lower bound we derive thus
also holds in that setting.




