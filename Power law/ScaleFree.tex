% !TEX root = Main.tex
\section{Scale Free  Graphs from Generative Models}\label{Sec:ScaleFree}
%In this section we treat the generative  BA model, conjectured to create power-law graphs\footnote{
%Bollobas~\cite{bollobas2003mathematical}(Sec. 6 and 7) proved that for a similar model vertices of degree smaller than $\sqrt[15]{n}$ in fact abide a power-law distribution. }.
The Barab{\'a}si-Albert (BA) model is a well-known generative model for power-law graphs that, roughly, grows a graph in a sequence of time steps by
inserting a single vertex at each step and attaching it to $m$ existing vertices with probability weighted by the degree of each existing vertex \cite{barabasi1999emergence}. The BA model
generates graphs that asymptotically have a power-law degree distribution ($\alpha = 3$) for low-degree nodes \cite{DBLP:journals/rsa/BollobasRST01}.
Graphs created by the BA model have low arboricity (the arboricity of a graph is the minimum number of spanning forests needed to cover its edges.)~\cite{goel2006bounded}; we use
that fact to prove the following highly efficient labeling scheme. 

%We present first a relation between the number of distinct (non-isomorphic) graphs in a family and the smallest label size possible.
%\begin{lemma}\label{Lemma:Relations}
%Let $\vert \mathcal{F}_n  \vert$ denote the number of non-isomorphic graphs in $\mathcal{F}_n$ and let $N$ be the smallest number of vertices of any induced universal graph for $\mathcal{F}_n$
%(In other words, any labeling scheme for $\mathcal{F}$ is of size at least $\log N$).
%Then $\log N - \log n \leq \log \vert  \mathcal{F}_n  \vert$.
%\end{lemma}


\begin{proposition}\label{Th:baLabeling}
The family of graphs generated by the BA model has an $O(m \log n)$ adjacency labeling scheme.
\end{proposition}

\begin{proof}
Let  $G=(V,E)$  be an $n$-vertex graph resulting by the construction  by the BA model with some parameter $m$ (starting from some graph $G_0 = (V_0,E_0)$ with $\vert V_0 \vert \ll n$).
While it is not known how to compute the   arboricity of a graph efficiently, it is possible in near-linear time to compute a partition of $G$ with  at most twice\footnote{More precisely, for any $\epsilon \in (0,1)$  there exist an $O(|E(G)| / \epsilon)$ algorithm~\cite{kowalik2006approximation} that computes such partition using at most $(1+ \epsilon)$ times more forests than the optimal.} the number of forests in comparison to the optimal~\cite{arikati1997efficient}.
We can thus decompose the graph to $2m$ forests in near linear time and label each forest using Alstrup and Rauhe's~\cite{Alstrup02} $\log n + O(\log^* n)$ labeling scheme for trees,  and achieve a $2m (\log n + O(\log^* n))$ labeling scheme for $G$.
\end{proof}

Note that if the encoder operates at the same time as the creation of the graph, Proposition \ref{Th:baLabeling} can be strengthened to yield an an $m \log n$ labeling scheme: simply store the  
 identifiers of the $m$ vertices attached with every vertex insertion.
Theorem~\ref{centralLowerBound} and Proposition \ref{Th:baLabeling} strongly suggest that, for each sufficiently large $n$, the number of  power-law graphs with $n$ vertices  is vastly larger than the number of graphs with $n$ vertices created by the  BA model.  In contrast, other generative models such as   Waxman~\cite{waxman1988routing}, N-level Hierarchical~\cite{calvert1997modeling}.
and Chung's~\cite{chung2006complex} (Chapter 3)  do not seem to have an obvious smaller label size than the one in Proposition~\ref{prop:labelingMain}.
