% !TEX root = Www.tex
\section{Distance labeling scheme}\label{Sec:Distance}
Chung and Lu \cite{chung2004average}  proved that the diameter of  power law random graphs \footnote{subject to mild conditions, and with $\alpha > 2$.}  is almost surely $\Theta(\log n)$ and the average distance is $O(\log \log n)$. 
In this section we demonstrate the usefulness of the fat/thin separation by introducing  a distance labeling scheme for small distances in power law graphs. 



The \emph{distance} between two nodes in an undirected graph is the length of the shortest path connecting the two nodes, if it exists, and $\infty$ if no such path exists.
Let $f : \mathbb{N} \longrightarrow \mathbb{N}$ be a map such that $f(n) \leq  -1$ for all $n$. An $f(n)$-\emph{distance labelling scheme} is a labelling
scheme such that, for any graph $G$, its decoder given labels $\mathcal{L}(u)$ and $\mathcal{L}(v)$ of two nodes $u$ and $v$
will output the distance between $u$ and $v$ if the distance is at most $f(\vert V(G) \vert)$, and output ``no'' if the distance
is  strictly greater than $f(\vert G \vert)$. If $f(n) = n-1$.
Labeling schemes for distance for sparse graphs have been recently investigated in~\cite{DBLP:journals/corr/AlstrupDKP15} and~\cite{DBLP:journals/corr/GawrychowskiKU15}.
The latter achieved the best bound known, namely $O(\frac{n}{D} \log D)$ where $D = (\log n)/(\log \frac{m+n}{n})$ and $m$ is the number of edges
in the graph.  
In addition to the lower bound of $\Omega(\sqrt{n})$ for adjacency given in the present paper (which trivially also a lower bound for distance), Gavoille et al.~\cite{gavoillea2004distance} proved a lower bound of  $\Omega(n^{3/2})$ on the total label size.

We now devise an $f(n)$-distance labelling scheme for $c$-sparse graphs,
works particular well (i.e., has shorter labels than any known labelling schemes) for small distances. As
all power-law graphs will in general be sparse, and power-law graphs in general
have very small expected distances, the labelling scheme should work well for practical purposes in power-law graphs.

\begin{lemma}\label{lem:sparse_small_dist}
Let $c > 0$. For any computable $f : \mathbb{N} \longrightarrow \mathbb{N}$ 
$f(n) \leq  -1$ for all $n$, there is an $f(n)$-distance labelling scheme for the family of $c$-sparse graphs
that assigns labels of length at most $O(n^{f(n)/(f(n) + 1)} \log f(n))$.
\end{lemma}

\begin{proof}
%As for adjacency labelling, the scheme is based on \emph{thin} and \emph{fat} nodes.
 Let $G$ be a $c$-sparse graph. Call a node of $G$ \emph{fat} if it has degree at least $n^{1/(f(n)+1)}$ and \emph{thin} otherwise.
The label of each node $v$ now contains (i) a table of distances to all fat nodes that are at distance at most $f(n)$ from $v$ and (ii) a table of distances to all thin nodes $w$ that are at most distance $f(n)$ away from $v$
where the shortest path between $v$ and $w$ does not pass through any fat node.
Clearly, as $f(n)$ is computable and distances in $G$ are computable, there is a computable encoder
assigning labels. Furthermore, as all nodes of $G$ are either thin or fat, it is clearly possible for an encoder to compute
all distances less than or equal to $f(n)$ between any pair of nodes. Note that as all distances we care for 
are bounded above by $f(n)$, each such distance can be stored using at most $\log f(n)$ bits.

As the sum of degrees of fat nodes are at most $2cn$ in a $c$-sparse graph, there can be at most
$$
\frac{2cn}{n^{\frac{1}{(f(n)+1}}} = 2cn^{1-\frac{1}{f(n)+1}} = 2cn^{\frac{f(n)}{f(n)+1}}
$$
fat nodes in $G$. Thus, a table of distances to all fat nodes takes up at most $O(n^{\frac{f(n)}{f(n)+1}} \log f(n))$ bits.

Similarly, for each node $v$ there are at most $(n^{1/(f(n)+1)})^{f(n)} = n^{f(n)/(f(n)+1)}$ nodes at distance at most $f(n)$ away from $v$ where the shortest path consists only of thin nodes.
 Hence, the associated table of distances takes up at most \\ $O(n^{f(n)/(f(n)+1)} \log n)$ bits.

In total, each label thus has size at most $O(n^{f(n)/(f(n)+1)} \log n)$ bits.
\end{proof}

For $f(n) = \log n$, Lemma \ref{lem:sparse_small_dist} yields a labelling scheme having label size
at most $O\left(n^{(\log n)/(1+ \log n)} \log\log n \right)$. Unsurprisingly, as we are only considering distances up
to $f(n)$, this label size is asymptotically smaller than for the labelling schemes working for all distances in sparse graphs, e.g. the largest label sizes of \cite{DBLP:journals/corr/GawrychowskiKU15} for sparse graphs is $O((n/\log n) \log\log n)$.
