% !TEX root = WWW.tex
\section{A distance labeling scheme}\label{Sec:Distance}
In this section we propose a distance labeling scheme for power law graphs. 

The \emph{distance} between two nodes in an undirected graph is the length of the shortest path connecting
the two nodes, if it exists, and $\infty$ if no such path exists.

Let $f : \mathbb{N} \longrightarrow \mathbb{N}$ be a map such that
$f(n) \leq  n -1$ for all $n$. An $f(n)$-\emph{distance labelling scheme} is a labelling
scheme such that, for any graph $G$, its decoder given labels $\mathcal{L}(u)$ and $\mathcal{L}(v)$ of two nodes $u$ and $v$
will output the distance between $u$ and $v$ if the distance is at most $f(\vert V(G) \vert)$, and output ``no'' if the distance
is  strictly greater than $f(\vert G \vert)$. If $f(n) = n-1$,
an $f(n)$distance labelling scheme is simply called a \emph{distance labelling scheme}.

For sparse graphs,
Alstrup et al.\ \cite{DBLP:journals/corr/AlstrupDKP15} obtain a distance labelling scheme with maximum label size
$O(\frac{n}{D} \log^2 D)$ where $D = (\log n)/(\log \frac{m+n}{n})$ and $m$ is the number of edges
in the graph.   Gawrychowski et al.\ obtain an upper bond of \cite{DBLP:journals/corr/GawrychowskiKU15}
$O(\frac{n}{D} \log D)$ with sub-linear decoding time. Few general results on lower bounds exist. The lower bound of $\Omega(\sqrt{n})$ for adjacency given in the present paper is trivially also a lower bound for distance; for total label size, the best known lower bound remains $\Omega(n^{3/2})$ as proved by Gavoille et al.~\cite{Gavoille2001}.

We now devise an $f(n)$-distance labelling scheme for $\PLBARG{\chi}{\alpha}$ working for any $\chi(n) \geq n^{1/(\alpha - 1 +f(n))}$.
The scheme has shorter labels than any known labelling schemes for small distances. As
the class of graphs with power law degree distribution is a subset of $\PLBARG{\chi}{\alpha}$, and as power-law graphs in general
have very small expected distances, the labelling scheme should work well for practical purposes in power-law graphs.

\begin{lemma}\label{lem:sparse_small_dist}
For any computable $f : \mathbb{N} \longrightarrow \mathbb{N}$ such that
$f(n) \leq n -1$ for all $n$, and for any $\chi(n) \geq n^{1/(alpha-1+f(n))}$ there is an $f(n)$-distance labelling scheme for $\PLBARG{\chi}{\alpha}$
that assigns labels of length at most $O(n^{f(n)/(f(n) + 1)} \log f(n))$.
\end{lemma}

\begin{proof}
As for adjacency labelling, the scheme is based on \emph{thin} and \emph{fat} nodes. Let $G$ be a graph
in $\PLBARG{\chi}{\alpha}$ Call
a node of $G$ \emph{fat} if it has degree at least $n^{1/(\alpha - 1 + f(n)}$ and \emph{thin} otherwise.
The label of each node $v$ now contains (i) a table of distances to all fat nodes (if the distance is more than $f(n)$, it is simply ignored), (ii) a table of distances to all thin nodes $w$ that are at most distance $f(n)$ away from $v$
where the shortest path between $v$ and $w$ does not pass through any fat node, and (iii) a single bit signifiying whether the node is fat or thin.
Clearly, as $f(n)$ is computable and distances in $G$ are computable, there is a computable encoder
assigning labels. A decoder can now compute the distance between any two nodes $u,v$ as follows:
If both $u$ or $v$ are fat, the distance can be directly read off part (i) of the label of any node. If at least one of $u$ 
and $v$ is fat, the distance can be read off part (i) of the label of the thin node. If both nodes are thin, the decoder
can check if the distance is in part (ii) of the label of either node; if the distance is not present, 
either the distance is strictly greater than $f(n)$, or the shortest path between $u$ and $v$ passes through
a fat node; in this case, the decoder may brute-force check the distances from $u$ and $v$ to each fat node,
and simply output the smallest sum of these two distances.

Furthermore, as all nodes of $G$ are either thin or fat, it is clearly possible for an encoder to compute
all distances less than or equal to $f(n)$ between any pair of nodes. Note that as all distances we care for 
are bounded above by $f(n)$, each such distance can be stored using at most $\log f(n)$ bits.

As $G = G(V,E)$ is in $\PLBARG{\chi}{\alpha}$, we have 
\begin{align*}
\sum_{i = \chi(n)}^{n-1} \vert V_i \vert
&\leq \sum_{i = n^{\frac{1}{\alpha - 1 + f(n)}}}^{n-1} \vert V_i \vert \leq C' \left( \frac{n}{\left( n^{\frac{1}{\alpha - 1 + f(n)}}\right)^{\alpha - 1}}\right)\\
&\leq C' n^{1 - (\alpha - 1)/{\alpha-1 + f(n)}} 
= C' n^{f(n)/(\alpha - 1 + f(n))}
\end{align*}
 Thus, a table of distances to all fat nodes takes up at most $O\left(n^{\frac{f(n)}{\alpha - 1 + f(n)}} \log f(n)\right)$ bits.

Similarly, for each node $v$ there are at most $\left(n^{1/(\alpha - 1 + f(n))}\right)^{f(n)} = n^{f(n)/(\alpha - 1 + f(n))}$ nodes at distance at most $f(n)$ away from $v$ where the shortest path consists only of thin nodes. Hence, the associated table of distances
takes up at most \\$O(n^{f(n)/(\alpha -1 + f(n))} \log n)$ bits.

In total, each label thus has size at most $O(n^{f(n)/(f(n)+1)} \log n)$ bits.
\end{proof}

For $f(n) = \log n$, Lemma \ref{lem:sparse_small_dist} yields a labelling scheme having label size
at most $O\left(n^{(\log n)/(\alpha - 1 + \log n)} \log\log n \right)$. Unsurprisingly, as we are only considering distances up
to $f(n)$, this label size is asymptotically smaller than for the labelling
schemes working for all distances in \emph{sparse} graphs, e.g. the largest label sizes of \cite{DBLP:journals/corr/GawrychowskiKU15} for sparse graphs is $O(n \frac{\log \log n}{ \log n})$.
For power law random graphs, Chung and Lu show in \cite{chung2004average} that, subject to mild conditions, the diameter of power law graphs with $\alpha > 2$ is almost surely $\Theta(\log n)$. We thus expect our labelling scheme to have
superior performance for such graphs.
