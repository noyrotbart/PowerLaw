% !TEX root = WWW.tex

\section{The Labeling Schemes}
\label{sec:lab_schem}
We now construct algorithms for labeling schemes for $c$-sparse graphs and for the family $\PLB$. Both labeling schemes partition vertices into \emph{thin} vertices which are of low degree and \emph{fat} vertices of high degree. The \emph{degree threshold} for the scheme is the lowest possible degree of a fat vertex. We start with $c$-sparse graphs.
\begin{theorem}\label{sparse-label}
There is a $\sqrt{2cn\log n} + 2\log n + 1$ labeling scheme for $\Sparse$.
\end{theorem}
\begin{proof}
Let $G=(V,E)$ be an $n$-vertex $c$-sparse graph. Let $\threshold{n}$ be the degree threshold for $n$-vertex graphs; we choose $\threshold{n}$ below. Let $k$ denote the number of fat vertices of $G$, and assign each fat vertex a unique identifier between $1$ and $k$. Each thin vertex is given a unique identifier between $k+1$ and $n$.

For a $v\in V$, the first part of the label $\la(v)$ is a single bit indicating whether $v$ is thin or fat followed by a string of $\log n$ bits representing its identifier. If $v$ is thin, the last part of $\la(v)$ is the concatenation of the identifiers of the neighbors of $v$. If $v$ is fat, the last part of $\la(v)$ is a \emph{fat bit string} of length $k$ where the $i$th bit is $1$ iff $v$ is incident to the (fat) vertex with identifier $i$.

Decoding a pair $(\la(u),\la(v))$ is now straightforward: if one of the vertices, say $u$, is thin, $u$ and $v$ are adjacent iff the identifier of $v$ is part of the label of $u$. If both $u$ and $v$ are fat then they are adjacent iff the $i$th bit of the fat bit string of $\la(u)$ is $1$ where $i$ is the identifier of $v$.
Both decoding processes can be computed in $O(\log n)$ time using standard assumptions.

Since $|E|\leq cn$, we have $k\leq 2cn/\threshold{n}$. A fat vertex thus has label size $1 + \log n + k\leq 1 + \log n + 2cn/\threshold{n}$ and a thin vertex has label size at most $1 + \log n + \threshold{n}\log n$. To minimize the maximum possible label size, we solve $2cn/x = x\log n$. Solving this gives $x = \sqrt{2cn/\log n}$ and setting $\threshold{n} = \lceil x\rceil$ gives a label size of at most $1 + \log n + (\sqrt{2cn/\log n} + 1)\log n\leq 1 + 2\log n + \sqrt{2cn\log n}$.
\end{proof}

%\begin{remark}\label{remark:FamilySize}
%It is easy to see that $f(n)$-sparse graphs enjoy a  $\sqrt{2f(n)} \log n$ %labeling scheme by setting the degree threshold to $\sqrt{2f(n)}$.
%In addition $c$-polysparse graphs enjoy a $n^{\frac{c}{2}} \log n$ labeling %scheme by setting the threshold to $n^{\frac{c}{2}}$.
%\end{remark}

By Proposition~\ref{prop:powerlawsparse}, graphs in $\PLC$ are sparse for $\alpha > 2$. This gives a label size of $O(\sqrt{n\log n})$ with the labeling scheme in Theorem~\ref{sparse-label}. We now show that this label can be significantly improved, by constructing a labeling scheme for $\PLB$ which contains $\PLC$.

\begin{theorem}\label{prop:labelingMain}
 There is a $\sqrt[\alpha]{C'n}(\log n)^{1 - 1/\alpha} + 2\log n + 1$ labeling scheme for $\PLB$.
\end{theorem}
\begin{proof}
The proof is very similar to that of Theorem~\ref{sparse-label}. We let $\threshold{n}$ denote the degree threshold. If we pick $\threshold{n}\geq \sqrt[\alpha]{n/\log n}$ then by Definition~\ref{def:general-family}  there are at most $C'n / \threshold{n}^{\alpha -1}$ fat vertices. Defining labels in the same way as in Theorem~\ref{sparse-label} gives a label size for thin vertices of at most $1 + \log n + \threshold{n}\log n$ and a label size for fat vertices of at most
$1 + \log n + C'n / \threshold{n}^{\alpha -1}$.
We minimize by solving
$x \log n = C'n / x^{\alpha -1}$, giving $x = \sqrt[\alpha]{C'n/\log n}$. Setting $\threshold{n} = \lceil x\rceil$ gives a label size of at most $\sqrt[\alpha]{C'n}(\log n)^{1 - 1/\alpha} + 2\log n + 1$.
\end{proof}
% !TEX root = WWW.tex
\subsection{A labeling scheme for random graphs}
There are schemes using randomness to ``grow''  graphs that, with high probability, have approximate power-law degree distribution for a range of degrees. One such scheme is the preferential attachment model \cite{barabasi1999emergence}. For graphs obtained from such models, their degree sequences are instead probability distributions. We now show that applying our labeling scheme for $\PLB$ to random graphs with the power law distribution results in a small expected worst-case label size.

Using the definition of Mitzenmacher \cite{mitzenmacher2004brief}, a random variable $X$ is said to have the \emph{power law} distribution (w.r.t.~$\alpha > 1$) if
\[
  Pr[X\geq x] \sim cx^{-\alpha+1},
\]
for a constant $c > 0$, i.e., $\lim_{x\to\infty}Pr[X\geq x]/cx^{-\alpha+1} = 1$.

Let $\epsilon > 0$ be fixed. Consider a graph $G$ picked from a family $\mathcal F$ of random graphs whose degree sequences have the power law distribution. Order the vertices of $G$ arbitrarily as $v_1,\ldots,v_n$. For $i=1,\ldots,n$, let indicator variable $X_i$ be $1$ iff $v_i$ has degree at least $d = \sqrt[\alpha]{n/\log n}$. There is a constant $N_0\in\mathbb N$ (depending on $\epsilon$) such that if $n\geq N_0$ then for all $i$,
\[
  E[X_i] = \mbox{Pr}[X_i = 1]\leq (1+\epsilon)cd^{-\alpha+1}.
\]

With the same labeling scheme as for $\PLB$ with degree threshold $\threshold{n} = d$, denote by $E_n$ the expected label size of an $n$-vertex graph from $\mathcal F$. Then for all $n\geq N_0$,
\begin{align*}
  E_n & = \sum_{x=0}^n \mbox{Pr}\left[\sum_{i=1}^n X_i = x\right]O((x+d\log n))\\
                       & = O\left(d\log n + E\left[\sum_{i=1}^n X_i\right]\right)\\
                       & = O\left(d\log n + \sum_{i=1}^n E[X_i]\right)\\
                       & = O\left(d\log n + nd^{-\alpha+1}\right)\\
                       & = O\left(\sqrt[\alpha]n(\log n)^{1-1/\alpha}\right).
\end{align*}
\begin{theorem}\label{th:random}
Let $\mathcal F$ be a family of graphs with degree sequences having the power law distribution w.r.t.~$\alpha > 1$. Then there is a labeling scheme for $\mathcal F$ such that the expected worst-case label size of any graph $G\in\mathcal F$ is $O(\sqrt[\alpha]n(\log n)^{1-1/\alpha})$ where $n$ is the number of vertices of $G$.
\end{theorem}

