\subsection{Labeling scheme for random graphs}
Graphs in $\PLB$ have a fixed degree sequence. However, certain graph generation models for power-law graphs are inherently random. For graphs obtained from such models, their degree sequences are instead probability distributions. In this section, we show that applying our labeling scheme for $\PLB$ to random graphs with the power law distribution, we get a good expected worst-case label size.

Using the definition of Mitzenmacher, a random variable $X$ is said to have the \emph{power law} distribution (w.r.t.~$\alpha > 1$) if
\[
  Pr[X\geq x] \sim cx^{-\alpha+1},
\]
for some constant $c > 0$, i.e., $\lim_{x\to\infty}Pr[X\geq x]/cx^{-\alpha+1} = 1$.

Let $\epsilon > 0$ be fixed. Consider a graph $G$ obtained from some random process leading to a power law distribution on its vertex degrees (e.g., obtained from the preferential attachment model). Order the vertices of $G$ as $v_1,\ldots,v_n$ where the ordering is independent of the random process leading to $G$. For $i=1,\ldots,n$, let indicator variable $X_i$ be $1$ iff $v_i$ has degree at least $d = \sqrt[\alpha]{n/\log n}$. There is a constant $N_0\in\mathbb N$ (depending on $\epsilon$) such that if $n\geq N_0$ then for each $i$,
\[
  E[X_i] = Pr[X_i = 1]\leq (1+\epsilon)cd^{-\alpha+1}.
\]

With the same labeling scheme as for $\PLB$ with degree threshold $d$, we get a labeling scheme with an expected label size of
\begin{align*}
  E[\mbox{label size}] & = \sum_{x=0}^n Pr\left[\sum_{i=1}^n X_i = x\right]O((x+d\log n))\\
                       & = O\left(d\log n + E\left[\sum_{i=1}^n X_i\right]\right)\\
                       & = O\left(d\log n + \sum_{i=1}^n E[X_i]\right)\\
                       & = O\left(d\log n + nd^{-\alpha+1}\right)\\
                       & = O\left(\sqrt[\alpha]n(\log n)^{1-1/\alpha}\right).
\end{align*}

