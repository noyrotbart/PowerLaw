% !TEX root = Www.tex

\section{Preliminaries}
Throughout the paper we consider $n$-vertex, undirected, finite graphs.
For real $c > 0$, a graph is $c$-\emph{sparse} if it has at most $cn$ edges and \emph{sparse} if it is $c$-sparse for some constant $c$. For $0 < c \leq n-1$, the set of $c$-sparse graphs with $n$ vertices is denoted by $\Sparse$.
If $\mathcal{F}$ is a set of graphs,  $\mathcal{F}_n$ denotes the subset of graphs in $\mathcal{F}$ having exactly $n$
vertices. The degree of a vertex $v$ in a graph is denoted by $\Delta(v)$, and for non-negative integers $k$,
the set of vertices in a graph $G$ of degree $k$  is denoted by $V_k$.
The length of a binary string $x \in \{ 0,1 \}^*$ is denoted by $\vert x \vert$.
%We  denote the concatenation of two binary strings $x,y$  by $x \circ y$.

Let  $\mathcal{F}$ be a set of graphs. An  \emph{adjacency labeling scheme} (from hereon just \emph{labeling scheme}) for  $\mathcal{G}$
is a pair consisting of an \emph{encoder} and a \emph{decoder}. The encoder is an algorithm that receives $G \in \mathcal{G}$ as input and outputs a bit string $\la(v) \in \{0,1\}^*$ called the \emph{label} of $v$. The decoder 
is an algorithm that receives any two labels $\la(v),\la(u)$ as input and outputs $\mathtt{true}$ if{f} $u$ and $v$
are adjacent in $G$ and $\mathtt{false}$ otherwise. Note that the graph $G$ is not an input to the decoder.
The \emph{size} of a labeling scheme is the map $\textrm{size} : \mathbb{N} \longrightarrow \mathbb{N}$
such that $\textrm{size}(n)$ is the maximum length of any label assigned by the encoder to any vertex in
any graph $G \in \mathcal{F}_n$. The \emph{degree distribution} of a graph $G = (V,E)$ is the mapping
 $\mathrm{ddist}_G (k) : \mathbb{N}_0 \longrightarrow \mathbb{Q}$
defined by $\mathrm{ddist}_G (k) := \frac{\vert V_k \vert} {n}$.


%As the literature on power-law graphs usually allows some noise and deviation from the ``pure'' power-law distribution,  when devising our labeling scheme we consider a larger class of graphs, $\PLB$, that allows both a constant
%factor deviation from each point on the curve $k\mapsto Ck^{-\alpha}$ and an arbitrarily large deviation for small values of $k$. The class $\PLB$ subsumes the above ``ideal'' class $\PL$, whence our upper bounds on the size of labeling schemes hold for both $\PLB$ and $\PL$. Conversely, our lower bounds are proved on a very restrictive subclass $\PLC$ of $\PL$ whence the lower bounds hold for both $\PLC$ and $\PL$.


