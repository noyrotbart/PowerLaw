\documentclass{article}

\usepackage{amssymb}
\usepackage{amsmath}
\usepackage{amsthm}
%\usepackage{theorem}

\newtheorem{definition}{Definition}
\newtheorem{proposition}{Proposition}
\newtheorem{lemma}{Lemma}
\newtheorem{conjecture}{Conjecture}
\theoremstyle{remark}
\newtheorem{remark}{Remark}

\newcommand{\la}{\ensuremath{ \mathcal{L}}}


\title{Adjacency labeling  schemes for sparse and power law graphs.}
\author{Noy Rotbart, Jakob Grue Simonsen and Christian Wulff Nielsen}

\begin{document}

\maketitle
\begin{abstract}
We devise adjacency labeling schemes for power-law graphs. These are a sub-family of sparse graphs that  received considerable attention in the literature in the last decade. We first find a lower bound for the size of the induced universal graphs required to each of the families.
We then  provide a constructive upper bound for these induced universal graphs using adjacency  labeling schemes of similar asymptotic size.
\end{abstract}
\section{Introduction}
Power-law graphs appear in vast number of places.
The number of nodes in power-law graphs seen in practice is of magnitude $~10^{12}$, labels of magnitude $\sqrt{10^{12}}$ is reasonable to handle given the current state of hardware. 
\subsection{Previous work}
A  graph  $G$ is universal, respectively induced universal to the graph family $\mathcal{F}$ if it contains all graphs in  $\mathcal{F}$  as subgraphs, respectively induced subgraphs.
The size of the smallest induced universal graph was first studied by  Moon~\cite{moon1965minimal}. Seen from labelling schemes, he showed a lower bound of $n/2$ on the label size for general graphs, and an upper bound of  $n/2+ \log n$. This gap was recently closed by Alstrup et al.~\cite{alstrup2014adjacency}.
Universal graphs for sparse graphs were investigated first by Babai et al.~\cite{babai1982graphs} and improved by Alon and Asodi~\cite{Alon2002universal}. 
\begin{remark}
Let $N$ be the smallest number of nodes in an induced universal graph $G=(V,E)$ for a family $\mathcal{F}$ of graphs each with $n$ vertices.
If this family contains no isomorphic graphs, then ${N}\choose{n}$ is clearly an upper bound on $\vert \mathcal{F} \vert$.
\end{remark}
%\subsection{Induced universal graphs and the size of the family}
%We attempt to prove the following connection:
%The size of the smallest induced universal graph is tightly coupled with the number of the non-isomorphic graphs in the family it represents.
%
%Let $N$ be the smallest number of nodes in the an induced universal graph $G=V,E$ for a family of graphs with $n$ vertices $\mathcal{F}$.
%If this family contains no isomorphic graphs, then ${N}\choose{n}$ is clearly an upper bound on $\vert \mathcal{F} \vert$.
%This bound is not tight as can be seen in Figure~\ref{}.
%%Now, suppose that $N$ is the smallest possible, and that there are less than  ${N}\choose{n}$ non-isomorphic graphs in  $\mathcal{F}$.
%%This means that at least one graph $G'=(V',E') \in \mathcal{F}$ can be embedded in $G$ in more than one way, such that the two sets of vertices it embeds to are different.  



\section{Preliminaries}
In the following we deal with $n$-vertex undirected finite and connected graphs.
A graph $G=(V,E)$ is \emph{sparse} if $\vert E \vert = O(n)$, and more precisely a graph with at most $cn$ edges is called a $c$-sparse graph.
We  denote the family of $c$-sparse graphs with $n$ nodes as   $\mathcal{S}_C$.
The degree of a vertex $v$ in a graph $G=(V,E)$ is denoted by $\Delta(v)$.
For every $1 \leq k \leq n$  the collection of vertices of degree $k$ is denoted $V_k$.
We begin by a definition relating to the fraction of vertices of a certain degree.
\begin{definition}
The \emph{degree distribution} of a graph $G = (V,E)$ is the mapping 
 $\mathrm{ddist}_G (k) : \mathbb{N}_0 \longrightarrow \mathbb{Q}$
defined by $\mathrm{ddist}_G (k) := 
\frac{\vert V_k \vert} {n}.$
\end{definition}

For socalled \emph{power-law graphs}  $\mathrm{ddist}_G (k) \sim C k^{-\alpha}$
for real numbers $C > 0$ and $\alpha > 1$. As $n \mathrm{ddist}_G (k)$ is a positive integer,
but $n C k^{-\alpha}$ is, in general, a non-integral real, the ``$\sim$'' entails some rounding
of $n C k^{-\alpha}$.

 
In the literature, a number of generative models that ``grow'' graphs whose degree distributions are, with high probability, asymptotically ``close''  to $\mathrm{ddist}_G (k) \sim C k^{-\alpha}$ have been proposed for various values of
$\alpha$, most prominently the Barabasi-Albert model \cite{} and \ldots 
 Common to these are that they work in discrete time-steps with each step
 involving some randomization (e.g., in the Barabasi-Albert model, each step introduces a fresh node that connects
 to a fixed number of existing edges with some probability dependent on the degrees of the existing nodes). 
 

 
We require that $k \mapsto C \frac{n}{k^\alpha}$ be a probability distribution. Hence, in particular
$C \sum_{k=1}^\infty i^{-\alpha} = 1$, and hence $C = 1/\sum_{k=1}^\infty i^{-\alpha} = 1/\zeta(\alpha)$ where
$\zeta$ is the Riemann zeta function.
Thus, $C$ is dictated by the choice of $\alpha$.
   

 \subsection{Proper power-law graphs}
 
 We define a class of \emph{proper} power law graphs where the number of vertices of degree $k$
 must be $C \frac{n}{k^{\alpha}}$ rounded either up or down and the number of vertices of degree $k$ is non-increasing
 with $k$ (note that the function $k \mapsto  C \frac{1}{k^{\alpha}}$ is strictly decreasing, so these demands
 are more lax than they could have been).
  

\begin{definition}
Let $\alpha > 1$ be a real number. We say that a graph  $G=(V,E)$ is an  $\alpha$-\emph{proper power-law graph} if (i) for every $1 \leq i \leq n$:  $\vert V_i \vert = n \cdot \mathrm{ddist}_G(k)\in \{\lfloor \frac{1}{\zeta(\alpha)}\frac{n}{i^{\alpha}} \rfloor, \lceil \frac{1}{\zeta(\alpha)} \frac{n}{i^{\alpha}} \rceil\}$, and (ii) for every $1 \leq i \leq n-1$: $\vert V_i \vert \geq \vert V_{i+1} \vert$. 
\end{definition}

\begin{proposition}\label{prop:maxnodeproper}
The maximum degree of a node in an $\alpha$-proper power-law graph is at most $\left(\frac{1}{\zeta(\alpha) (\alpha - 1)} + 2\right) \sqrt[\alpha]{n} + 2$.
\end{proposition}

\begin{proof}
Let $n > 0$ be an arbitrary integer and let $k' \triangleq \lceil \sqrt[\alpha]{n} \rceil$. 
Furthermore, let $S_{k'} = \sum_{i=1}^{k'} |V_i|$, that is $S_{k'}$ is the number of nodes of degree at most $k'$.

Let $S^{-}_{k'} = \sum_{i=1}^{k'} \lfloor \frac{n}{\zeta(\alpha)} i^{-\alpha}\rfloor$. Then,
$S_{k'} \geq S^{-}_{k'}$. We now bound $S^{-}_{k'}$ from below.
For every $i$ with $1 \leq i \leq k'$ we have $\lfloor \frac{n}{\zeta(\alpha)} i^{-\alpha}\rfloor + 1\geq \frac{1}{\zeta(\alpha)} x^{-\alpha}$, and hence
\begin{align*}
S^{-}_{k'} + k' = &\sum_{i=1}^{k'} \left(\left\lfloor \frac{n}{\zeta(\alpha)} i^{-\alpha}\right\rfloor + 1\right) \geq  \sum_{i=1}^{k'} \frac{n}{\zeta(\alpha)} i^{- \alpha}  = \frac{n}{\zeta(\alpha)} \sum_{i=1}^{k'} i^{-\alpha} \geq \\
&n \left(1 - \frac{1}{\zeta(\alpha)} \sum_{i=k'+1}^{\infty} i^{-\alpha} \right) 
 \geq n \left( 1 - \frac{1}{\zeta(\alpha)}\int_{k'}^\infty x^{-\alpha} dx \right) = \\ 
 &n \left( 1 - \frac{1}{\zeta(\alpha)} \left[ \frac{1}{\alpha - 1} x^{-\alpha + 1}\right]_{\infty}^{k'}\right) = 
n \left( 1 - \frac{1}{\zeta(\alpha) (\alpha - 1)} \left( \lceil n^{\frac{1}{\alpha}} \rceil \right)^{-\alpha + 1}\right) \geq \\
 &n \left( 1 -  \frac{1}{\zeta(\alpha) (\alpha - 1)} \left( n^{\frac{1}{\alpha}} \right)^{-\alpha + 1} \right) = n - \frac{n}{\zeta(\alpha) (\alpha - 1)}n^{-1+\frac{1}{\alpha}} = \\
&n - \frac{1}{\zeta(\alpha) (\alpha - 1)}\sqrt[\alpha]{n}
\end{align*}
Thus, $S_{k'} \geq S^{-}_{k'} \geq n - \frac{1}{\zeta(\alpha) (\alpha - 1)}\sqrt[\alpha]{n} - \lceil \sqrt[\alpha]{n} \rceil$

As for every $1 \leq i \leq n-1$: $\vert V_i \vert \geq \vert V_{i+1} \vert$,
there are thus at most $1/(\zeta(\alpha) (\alpha - 1)) \sqrt[\alpha]{n} + \lceil \sqrt[\alpha]{n} \rceil$ nodes of degree strictly more than $k' = \lceil \sqrt[\alpha]{n} \rceil$. Hence, the maximum degree of any $\alpha$-proper power-law graph is at most $\left(\frac{1}{\zeta(\alpha) (\alpha - 1)} + 2\right) \sqrt[\alpha]{n} + 2$.
\end{proof}

\begin{proposition}
Let $C > 0$ and $\alpha > 2$. Then, any $\alpha$-proper power-law graph is sparse.
\end{proposition}

\begin{proof}
By Proposition \ref{prop:maxnodeproper}, the maximum degree of a node in an $\alpha$-proper power-law
graph is at most $k' \triangleq \left(\frac{1}{\zeta(\alpha) (\alpha - 1)} + 2\right) \sqrt[\alpha]{n} + 2$, whence
the total number of edges is at most $\frac{1}{2}\sum_{i=1}^{k'} k V_k$. By definition,
$V_k \leq \lceil \frac{1}{\zeta(\alpha)}\frac{n}{k^n}\rceil \leq \frac{1}{\zeta(\alpha)} \frac{n}{k^{\alpha}} + 1$, and thus
\begin{align*}
\frac{1}{2}\sum_{i=1}^{k'} k V_k &\leq \frac{1}{2}\sum_{i=1}^{k'} k \left( \frac{1}{\zeta(\alpha)} \frac{n}{k^{\alpha}} + 1 \right)
 \leq \frac{k'}{2} + \frac{n}{\zeta(\alpha)} \sum_{k=1}^{\infty} k^{-\alpha+1} \\
&\leq \left(\frac{1}{2\zeta(\alpha) (\alpha - 1)} + 1\right) \sqrt[\alpha]{n} + 1 + \frac{n \zeta(\alpha - 1)}{\zeta(\alpha)} 
\end{align*}
\end{proof}

FIXME! JGS: I don't know what happens at $1 \leq \alpha \leq 2$.




%The literature is not very clear on when to use $[\cdot]$, $\lfloor \cdot \rfloor$ or other rounding --- most papers assume a little slack,
%and it is common just to see ``$\mathrm{ddist}_G(k) \sim k^{-\alpha}$''\footnote{Also: when truncating the probability mass of a distribution with infinite support (e.g., power-law distributions), the excess probability mass needs to be accounted for. If we ever write a paper on this, we need to do this more formally than almost all existing papers.}. Furthermore,
%the degree distribution is sometimes only assumed to ``kick in'' for sufficiently large values of $k$.
%Power-law graphs are also known as ``scale-free networks'', or ``graphs with a fat-tailed degree distribution''.

%An unbelievable amount of literature has been written about power law graphs, almost all of it bad. A very large
%set of phenomena  that ``naturally'' involve graphs (protein networks, internet AS-level graphs, Facebook friends, \ldots) have been modelled more-or-less accurately by power-law graphs\footnote{I think most of it is crap --- ask Casper.}. The typical fit of $C k^{-\alpha}$ results in $1 < \alpha < 2$.

%We also ignore rounding and remark the two following equalities.
%\begin{enumerate}
%\item   $\mathrm{ddist}_G(k) = \frac{C}{k^{\alpha}}$  
%\item $\vert V_k \vert = \frac{nC}{k^{\alpha}}$.
%\end{enumerate}



\subsection{Approximate power-law graphs}
 

 \begin{definition}\label{def:approximate}
 Let $\alpha > 1$ and $\epsilon > 0$ be real numbers.
 An undirected graph $(V,E)$ is said to be an $\epsilon$-\emph{approximate} $\alpha$-\emph{power-law graph}
 if, for all $1 \leq k \leq n = \vert V \vert$, we have $\left\vert \frac{V_k}{n} - \frac{1}{\zeta(\alpha)} k^{-\alpha} \right\vert < \epsilon \frac{1}{\zeta(\alpha)} k^{-\alpha}$.
 \end{definition}
 
 That is, for an $\epsilon$-approximate $\alpha$-power-law graph, the probability that node has degree $k$ differs from
 a power law with exponent $\alpha$ by strictly less than $\epsilon$ weighted by the ``pure'' power-law probability
 that the node has degree $k$.
 
 FIXME: Perhaps a drawing of a histogram is appropriate. 
 
 As shown by Bollob{\`a}s et al.\ \cite{}, the Barabasi-Albert model will, for each $\epsilon$,
 produce graphs that with probability $1$ for all sufficiently large $n$ will be $\epsilon$-approximate
 $3$-power-law graphs (but only for the range $1 \leq k < \sqrt[15]{n}$; see the definition of ``$\epsilon$-approximate
 $\alpha$-power-law graph with cutoff'' below.).
  
We require the graphs to have a degree distribution that is a good approximation to a power-law for
\emph{all} $1 \leq k \leq n-1$. We stress that the adjacency labelling schemes we devise in Section \ldots work with the with upper bounds even if nodes above a certain degree threshold (e.g. $k \geq \sqrt[15]{n}$) are ignored (cf.\ the result by Bollob{\`a}s et al.).
  
\begin{proposition}
Let $\alpha > 1$ and $\epsilon > 0$ be real numbers. For all $n \geq \zeta(\alpha) k^{\alpha}/\epsilon$
there is an $\epsilon$-approximate $\alpha$-powerlaw graph with $n$ nodes.
 \end{proposition} 
  
\begin{proof}
 TBD
\end{proof} 
  
 FIXME! JGS: the lower bound on $n$ above doesn't matter. The important thing is that there is such a bound. 
  
  
 \begin{lemma}\label{lem:approximate_fewlarge}
 Let $\alpha > 1$ and $\epsilon > 0$ be real numbers. 
 The maximum degree of a node in an $\epsilon$-approximate $\alpha$-power-law graph is at most
$\left\lfloor \sqrt[\alpha]{\frac{(1+\epsilon)n}{\zeta(\alpha)}} \right\rfloor$. 
 \end{lemma}
 
 FIXME! JGS: There is a silly special case above where $\sqrt[\alpha]{n}$ is an integer where the formulation needs
 to be changed. Will just engender more notational confusion :-/
 
 \begin{proof}
The assumption that  $\vert \frac{V_k}{n} - \frac{1}{\zeta(\alpha)} k^{-\alpha} \vert < \frac{1}{\zeta(\alpha)} k^{-\alpha}\epsilon$ implies that $V_k/n - \epsilon\frac{1}{\zeta(\alpha)k^{-\alpha}} \leq \frac{1}{\zeta(\alpha)k^{-\alpha}}$, hence that
$V_k \leq \frac{n}{\zeta(\alpha)}k^{-\alpha} (1 + \epsilon)$. But
$\frac{n}{\zeta(\alpha)}k^{-\alpha} (1 + \epsilon) < 1 if{f} k > \sqrt[\alpha]{\frac{(1+\epsilon)n}{\zeta(\alpha)}}$,
whence $k > \sqrt[\alpha]{\frac{(1+\epsilon)n}{\zeta(\alpha)}}$ implies $V_k = 0$ (as $V_k$ is a non-negative integer).
 \end{proof}
 

 
 \begin{proposition}
 Let $\alpha > 2$ and $\epsilon > 0$ be real numbers. Any $\epsilon$-approximate $\alpha$-power-law graph
 is $\frac{(1+\epsilon)\zeta(\alpha-1)}{2\zeta(\alpha)}$-sparse. 
 \end{proposition}
 
 \begin{proof}
By Lemma \ref{lem:approximate_fewlarge}, the maximum degree of a node in an $\epsilon$-approximate $\alpha$-power-law graph is at most $k' \triangleq \left\lfloor \sqrt[\alpha]{\frac{(1+\epsilon)n}{\zeta(\alpha)}} \right\rfloor$. Hence, the total
 number of edges is at most $\frac{1}{2}\sum_{k=1}^{k'} k V_k$.
 
 As in the proof of Lemma \ref{lem:approximate_fewlarge}, observe that $V_k \leq \frac{n}{\zeta(\alpha)}k^{-\alpha} (1 + \epsilon)$, we hence have
 \begin{align*}
 \frac{1}{2}\sum_{k=1}^{k'} k V_k &\leq \frac{n(1+\epsilon)}{2\zeta(\alpha)} \sum_{k=1}^{k'} k^{-\alpha+1} 
 \leq  \frac{n(1+\epsilon)}{2\zeta(\alpha)} \sum_{k=1}^{\infty} k^{-\alpha+1}  
\leq \frac{n(1+\epsilon)\zeta(\alpha-1)}{2\zeta(\alpha)} 
 \end{align*}
 As $\zeta(\alpha - 1)$ is well-defined for $\alpha > 2$, the result follows.
 \end{proof}
 
 FIXME: JGS has no idea what happens for $1 < \alpha \leq 2$. Conceivably, the graphs could be non-sparse.
 
 \subsection{Approximate power-law graphs with cutoff}

Generative models may fail to generate graphs whose degree distributions follow power-laws exactly,
cf.\ the result by Bollob{\`a}s et al.\ \cite{} mentioned above. This is the motivation for the following definition.

\begin{definition}\label{def:approximate_with_cutoff}
Let $\alpha > 1$ and $\epsilon > 0$ be real numbers and let $h : \mathbb{N} \longrightarrow \mathbb{N}$
be a non-decreasing, unbounded function with $h(n) \leq n$.
 An undirected graph $(V,E)$ is said to be an $\epsilon$-\emph{approximate} $\alpha$-\emph{power-law graph with cutoff} $h$
 if, for all $1 \leq k \leq h(n) = \vert V \vert$, we have $\left\vert \frac{V_k}{n} - \frac{1}{\zeta(\alpha)} k^{-\alpha} \right\vert < \epsilon \frac{1}{\zeta(\alpha)} k^{-\alpha}$.
\end{definition}

The difference between Definitions \ref{def:approximate} and \ref{def:approximate_with_cutoff} is that
a graph with cutoff with $n$ nodes need only be approximate power-law below the ``cutoff'' of $h(n)$.

For specific $\epsilon,\alpha$ and $h$, for example, the $h(n) = \sqrt[15]{n}$ of Bollob{\`a}s et al.\ \cite{}, one may consider the family of all $\epsilon$-approximate $\alpha$-power-law graphs with cutoff $h$. The below
result shall later be key to the observation that such families in general do not have short labelling schemes
for adjacency.

\begin{lemma}
Let $\alpha > 1$ and $\epsilon > 0$ be real numbers and let $h : \mathbb{N} \longrightarrow \mathbb{N}$
be a non-decreasing, unbounded function with $h(n) \leq n$. Then, for every natural number $m$ and every undirected graph $G$ with $n - h(m) - ?$ nodes there is an $\epsilon$-approximate $\alpha$-power-law graph with cutoff $h$ having $blah$ nodes that contains the complement $G$ as a subgraph.
\end{lemma}

\begin{proof}
TBD
\end{proof}







\section{The Labeling Schemes}
We first handle sparse graphs.
\begin{proposition}\label{sparse-label}
There exist a $\sqrt {2cn} \log n+ \log n$ adjacency labeling scheme for $\mathcal{S}_c$.
\end{proposition}
\begin{proof}
Let $G=(V,E)$ be a $c$-sparse graph.
We first assign each vertex $v \in V$ a unique identifier $ID(v)$, using $\log n$ bits.
A vertex of degree at least $\sqrt{2cn}$ is called \emph{fat} and \emph{thin} otherwise.
From hereon, we use the terminology degree threshold to describe the value separating these two groups.
The first bit of $\la(v), v \in V$ is set to  zero if $v$ is fat and to  one if it is thin.
Since there are at most $2cn$ edges, the number of fat vertices  is at most $\sqrt{2cn}$.
Let $(u,v)$ be an edge in $G$ such that $ID(u)<ID(v)$. 
If $u$ and $v$ are both either thin or fat $ID(v)$ will appear in $\la(u)$ and vice versa.
If $u$ is fat and $v$ is thin, $ID(u)$ will appear in $ID(v)$.
Since there are at most $\sqrt{2cn}$ fat vertices, the size of the largest label is bounded by $\sqrt{2cn} \log n +\log n$.
Similarly, thin vertices enjoy the same label size as they have at degree at most $\sqrt{2cn}$.
Decoding the label is now obvious, and will take $O(\sqrt{n})$ operations.
\end{proof}

\begin{remark}
%% Christian's remark
%Considering $f(n)$-sparse graphs\footnote{Graphs with $n$ vertices and  $f(n)$ edges}, and a degree threshold $g(n)$, 
%the size of a thin label size is $g(n) \log n$, and the number of fat nodes is at most $ 2n\frac{f(n)}{g(n)}$.
%In this case, fat nodes have label size $2n\frac{f(n)}{g(n)}\log n$.
% Setting $g(n)\log n = 2n\frac{f(n)}{g(n)}\log n$, we have  a label size of $g(n)\log n = \sqrt{2f(n)n} \log n$. 
%This matches (ignoring logarithms) the bounds known for sparse and for dense graphs.
It is easy to see that $f(n)$-sparse graphs\footnote{Graphs with $n$ vertices and  $f(n)$ edges}
enjoy a  $\sqrt{2f(n)n} \log n$ labeling scheme by setting the degree threshold to $\sqrt{2f(n)n}$.
\end{remark}
%\begin{remark}
%The labeling scheme in Proposition~\ref{sparse-label} is valid for all $1 \leq c \leq n$. Moreover, it is favourable to Alstrup et al.'s labeling scheme for 
%$1 \leq c \leq \frac{\sqrt{n}}{\log n}$
%\end{remark}
Recall that  $\mathcal{P}_{C,\alpha} \in \mathcal{S}_{2C}$ when $\alpha \geq 2$.  This yields a $\sqrt{4C n} \log n$ labeling scheme for $\mathcal{P}_{C,\alpha}$.
We now show that this label can be significantly improved. To do so we first need to account for the number of vertices of degree at least $k$ for any $1\leq k \leq n$.
\begin{proposition}\label{prop:smallfraction}
Let $f : \mathbb{N} \longrightarrow \mathbb{N}$ be a mapping such that
$f(n) = o(n)$. Let $C > 0$ and $\alpha > 1$ be real numbers. Then
there is $N \in \mathbb{N}$ such that if $G$ is a power-law graph with $\mathrm{ddist}_G(k) = C k^{-\alpha}$
and at least $N$ vertices, then the fraction of vertices in $G$ with degree at least $f(n) + 1$
is bounded above by $O(f(n)^{-(\alpha-1)})$.
\end{proposition}

\begin{proof}
For $1 \leq j \leq n - 1$, the fraction of vertices of degree at least $j$
is $C\sum_{i=j}^{n-1} 1/i^{\alpha}.$
Note that $d/dx (Cx^{-\alpha}) =  -\alpha C x^{-(\alpha + 1)} < 0$
and thus  for all $j \geq 1$ we have $\int_{j}^{j+1} Cx^{-\alpha} dx > C(j+1)^{-\alpha}$.
Hence, the fraction of vertices of degree at least $j+1$ is at most
$$
\int_j^{n-1} C x^{-\alpha} dx = \left[ \frac{C}{-(\alpha - 1)} x^{-(\alpha-1)} \right]_j^{n-1}
= \frac{C}{\alpha - 1}\left(j^{-(\alpha - 1)} - (n-1)^{-(\alpha -1 )} \right)
$$

In particular, the fraction of vertices of degree at least $f(n) + 1$ is at most
 $\frac{C}{\alpha - 1}\left(f(n)^{-(\alpha - 1)} - (n-1)^{-(\alpha -1 )} \right)
\leq \frac{C}{\alpha - 1} f(n)^{-(\alpha - 1)} = O(f(n)^{-(\alpha - 1)})$.
\end{proof}
%%%




\begin{lemma}\label{lem:silly_bound}
Let $f : \mathbb{N} \longrightarrow \mathbb{N}$ be a computable mapping such that $f(n) = o(n)$
and let $C > 0, \alpha > 1$ be real numbers. Then the family of power-law graphs
with $\mathrm{ddist}_G(k) = C k^{-\alpha}$ has a adjacency labeling scheme
such that for all sufficiently large $n$, the maximum size of a label
is bounded above by:
$$\log n +\max(f(n) \log n   ,  O(n / f(n)^{\alpha -1})).$$
\end{lemma}

\begin{proof}
We set the degree threshold at $f(n)$. By that we mean that   a vertex $v \in V$ is \emph{small} if $\Delta(v) \leq f(n)$ and \emph{large} otherwise.
By Proposition \ref{prop:smallfraction} there are at most $c'n / f(n)^{\alpha -1}$ large  vertices for some $c'$.
We assign each vertex a unique identifier from $1 \dots n$, such that the large vertices are assigned the last $c' n / f(n)^{\alpha -1}$  identifiers according to their degree in non-decreasing order.

The label of a small vertex consists of two parts: its unique identifier and a list of the identifiers of all its neighbors.
The size of such label  is thus at  most  $(f(n)+1) \log n$.
The label of a large vertex consists also of two parts: a unique identifier and a bit string of length $c' \cdot  n \cdot f(n)^{-(\alpha -1)}$ such that  position $i$ is $1$ in this bit string if the vertex 
is adjacent to the $i$'th largest large vertex.
The size of a such  label is  at most $c' \cdot  n / f(n)^{\alpha +1}+ \log n.$

Let $\la(v), \la(w)$ be two labels assigned by our suggested decoder to  vertices $v,w \in V$.
 If $v$ and $w$ are both small or both large, then there is an edge from $v$ to $w$ if and only if $w$ is listed in the label of $v$ and vice versa.
 Assume w.l.o.g that $v$ is small and $w$ is large, then there is an edge from $v$ to $w$ if and only if $w$ is listed in the label of $w$
\end{proof}

From Lemma~\ref{lem:silly_bound} it follows that the smallest label size is attained when
$ f(n) \log n = c'n / f(n)^{\alpha -1}$ for some constant $c'$.
We rearrange the equation and get
$ f(n)^{\alpha} = c'n $ and thus the optimum label size occurs when $f(n) = \sqrt[\alpha]{c'n}$, and the resulting label size is  $O( \sqrt[\alpha]{n} \log n)$.
 
This also means that there are $O(\sqrt[\alpha]n)$ large vertices, which correlates nicely with the following fact.
The largest degree in an $(\alpha,C)$-power law graph is bounded by $O(\sqrt[\alpha]n)$.
To see this observe that the number of vertices of degree $k$ must  to be at least $1$. Thus 
$ nC\frac{1}{k^\alpha} \geq 1$, which implies that $k \leq \sqrt[\alpha]{nC}$ for some constant $C$.



\begin{conjecture}
Any family of graphs such that $\mathrm{ddist}(k)$ has ``high'' positive skewness will have labeling schemes for adjacency with
sublinar maximum labeling size. A reasonable way forward would be to consider the third moment of some standard distributions
and see what happens.
\end{conjecture}





\section{Lower Bounds}
We begin this section by showing that the upper bounds achieved for sparse graphs are fairly close to the best possible.
By Moon~\cite{moon1965minimal} it follows  that any adjacency labeling scheme for general graphs requires at least $\lfloor n/2 \rfloor$ bits.
For brevity, we assume now that $n$ is an even number.
We present the following extension, due to Spinard~\cite{spinrad2003efficient}.
\begin{proposition}
Any adjacency labeling scheme for $c$-sparse graphs requires  labels of size strictly larger than $\frac{\sqrt{n}}{2 \sqrt{c}}$ bits.
\end{proposition}
\begin{proof}
Assume for contradiction that there exist a labeling scheme for adjacency assigning labels of size strictly less than $\frac{\sqrt{n}}{2 \sqrt{c}}$.
Let $G$ be an $n$-vertices graph. Let $G'$ be the graph resulting by adding $\frac{n(n-1)}{c}$ isolated vertices to $G$, and note that now $G'$ is $c-sparse$. The graph $G$ is an induced subgraph of  $G'$.
It now follows that the nodes of $G$ have adjacency labels of size less than 
$ \frac{\sqrt{n^2/c}}{2 \sqrt{c}}= n/2$ bits. As $G$ was an arbitrary graph, we obtain a contradiction.
\end{proof}

\subsection{Lower bounds for $\mathcal{P}_{C,\alpha}$.}
We now show that a similar lower bound can be attained for power-law graphs where $\alpha>2$. To do so, we first must argue that  constructing power law graphs in  this fashion  is  possible.

A \emph{degree sequence} is a sequence of  integers $d_1 \dots d_n$ such that $0 \leq d_i \leq n-1$.
We denote $v_1 \dots v_n$ the vertices the vertices of a graph $G=(V,E)$ in non-increasing order according to their degree.
We  denote $v_i \in V$ the $i$'th vertex in this ordering.
The \emph{degree sequence} of $G$ is $d_1 \geq d_2 \dots \geq d_n$ where $d_i$ is the degree of $v_i$.
While every graph has a degree sequence, not all degree sequences have graphs.
We say that a degree sequence is \emph{realizable} if it has a corresponding graph. 
The \emph{Erd\"{o}s-Gallai theorem} \cite{erdos1960graphs} states the following:
A degree sequence $d_1 \dots d_N$ is realizable if and only if:
\begin{enumerate}
\item {For every $1 \leq k \leq N-1$:   $\sum_{i=1}^k d_i \leq k(k-1)+ \sum_{i=k+1}^N \min \{k, d_i \}.$}
\item {$\sum_{i=1}^n d_i$ is even.}
\end{enumerate}

We now show the following:
\begin{lemma}\label{power-law-is-realizable}
Any degree sequence $d_1 \geq d_2 \dots \geq d_n$   that abides to  an $(\alpha,C)$ power-law distribution  is realizable.
\end{lemma}
\begin{proof}
First, since graphs in $\mathcal{P}_{C,\alpha}$ are sparse, for every $k> O(\sqrt[\alpha] n)$ the condition holds trivially by observing that $k(k+1)> O(n)$.
The maximum degree of a vertex in a power law graph is $O(\sqrt[\alpha] n)$.
When   $k\leq O(\sqrt[\alpha] n)$ we have that:
$$\sum_{i=1}^k d_i < O(\sqrt[\alpha]n \sqrt[\alpha]n) = O(n^{2/\alpha}) = o(n)$$
since  $\alpha>2$.
It is easy to see that in this case $\sum_{i=k+1}^n \min\{k,d_i\} =O(n)$, and thus the inequality holds.

Finally, if  $\sum_{i=1}^n d_i$ is not even we add a single edge to the graph.
\end{proof}

\begin{lemma}\label{G-is-powerlaw}
Any $n$ vertices graph $G=(V,E)$ can be extended to an $N=cn^{\alpha+1}$ vertices graph $G'=(V',E')$ where $G' \in \mathcal{P}_{C,\alpha}$, such that $G$ is an induced subgraph of $G'$.
\end{lemma}
\begin{proof}
The number of vertices with degree at least $n$ in $G'$  is $O(\frac{N}{n^\alpha}) = O(n)$.
By Lemma~\ref{power-law-is-realizable}, we can create $G'$
 by first inserting the degree sequence corresponding the vertices of  $G$, and then complete the power lower distribution naively.
 
\end{proof}

We can now conclude with our lower bound:
\begin{proposition}\label{prop:lower-bound-powerlaw}
Any adjacency labeling scheme for  $\mathcal{P}_{C,\alpha}$ requires  least $O(\sqrt[\alpha+1]{n})$ bits.
\end{proposition}

\begin{proof}
Assume for contradiction that there exist a labeling scheme for adjacency assigning labels of size at most $o(\sqrt[\alpha+1]{n}/c)$ bits. 
Let $G$ be a graph of $n$-vertices. 
We now construct the graph $G'$ as described in Lemma~\ref{G-is-powerlaw}.
 The graph $G$ can be reconstructed by an adjacency labeling scheme for adjacency for $G'$ using only the labels of vertices belonging to $G$ inside $G'$. By the assumption, there is thus a labeling of $G'$ using
$o(\sqrt[\alpha+1]{n^{\alpha+1}}) = o(n)$ bits. As G is an  arbitrary graph, we obtain a contradiction.



\subsection{Constructive models and the implication of the lower bound.}
The two most used power-law constructors are Waxman~\cite{} and N-level Hierarchical~\cite{}.
The Barabasi-Albert model generates power-law graphs, such that given a parameter $m$, vertices are inserted to an initially empty graph and attached to at most $m$ existing vertices according to a power-law distribution.
We now prove that  the  constructions  by Waxman and Barabasi-Albert  can not possibly construct all power-law graphs.

It is easy to see that  graphs constructed by the Barabasi-Albert model  has an $m \log n$ adjacency labeling scheme.
Upon vertex insertion, simply store the identifiers of all vertices attached.
  This along with proposition~\ref{prop:lower-bound-powerlaw} suggests that there are  a lot more power-law graphs than ones that can be created by  preferential attachment, as in the Barabasi-Albert model.

\end{proof}



 \bibliographystyle{abbrv}
\bibliography{lit}


\end{document}
