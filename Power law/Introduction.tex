% !TEX root = WWW.tex


\section{Introduction}
A body of work on large real-world networks  deals with the difficulties of storing them.
Compression techniques~\cite{boldi2004webgraph,boldi2011layered}, as  well as the dissemination  of these networks underlying graphs over several machines~\cite{gonzalez2012powergraph, stanton2012streaming} are some of the ideas suggested to treat this task.
An additional  approach to solve these difficulties is to disseminate the structural information of the graph to its vertices.
 Known as peer-to-peer, it  allows for an inference of the graph's local topology  using only local information stored in each vertex without using costly access to large, global data structures. 
As mentioned by   Buchegger et al.~\cite{buchegger2009peerson}  peer-to-peer  networks are particularly useful to address privacy concerns.

To that end, a useful tool is the notion of a \emph{labeling scheme}: an algorithm that assigns a bit string--a \emph{label}--to each vertex so that a query between any two vertices can be deduced solely from their respective labels. 
Labeling schemes are extremely well studied topic \cite{}, under the objective of  minimizing the \emph{maximum label size}: the maximum number of bits used in a label of any vertex. Among applications for them are   XML search engines~\cite{cohen2010labeling}, mapping services~\cite{abraham2011hub} and internet routing~\cite{krioukov2004compact}.

%Similarly to~\cite{}, we  restrict our attention to the atomic operation of \emph{adjacency} queries. 
Adjacency labeling schemes for general graphs  require a label size of $n/2+O(1)$~\cite{moon1965minimal, alstrup2014adjacency}.
Trees, planar graphs, and bounded degree graphs, on the other hand,   enjoy labels of logarithmic size~\cite{Alstrup02, gavoille2007shorter, adjiashvili2014labeling}. 
To the best of our knowledge, we are the first to study adjacency labeling schemes for classes of graphs whose statistical properties--in particular their \emph{degree distribution}--more closely resemble that of real-world networks.

One class of graphs extensively used for modelling real-world networks is \emph{power-law graphs}: roughly, $n$-vertex graphs where the number of vertices of degree $k$ is proportional to $n/k^{\alpha}$ for some positive $\alpha$. Power-law graphs (also called scale-free graphs in the literature) have been used, e.g., to model the Internet AS-level graph \cite{DBLP:journals/ton/SiganosFFF03,DBLP:conf/podc/AkellaCKS03}, and many other types of network (see, e.g., \cite{mitzenmacher2004brief,clauset2009power} for overviews). 
The adequacy of fit of power-law graph models to actual data, as well as the empirical correctness of the conjectured mechanisms giving rise to power-law behaviour, have been subject to criticism (see, e.g., \cite{DBLP:journals/jacm/AchlioptasCKM09,clauset2009power}). 
In spite of such criticism, and because their degree distribution affords a reasonable approximation of the degree distribution of many networks, the class of power-law graphs remains a popular tool in network modelling whose statistical behaviour is well-understood: e.g., for power-law graphs with $2<\alpha<3$, the range most often seen in the modelling of real-world networks \cite{clauset2009power}, it is known that with high probability the average distance between any two vertices is  $O(\log \log n)$, the diameter is $O(\log n)$ and there exists a dense subgraph of $n^{c/\log \log n}$ vertices~\cite{chung2004average}. 

Routing labeling schemes for power-law graphs  have been investigated by Brady and Cowen~\cite{brady2006compact}, and by Chen et al.~\cite{chen2012compact}. Labeling schemes for other properties than adjacency have been investigated for various classes of graphs, e.g., distance~\cite{gavoillea2004distance}, and flow~\cite{katz2004labeling}. 
Dynamic labeling schemes were studied by Korman and Peleg~\cite{korman2007compact,Korman07,korman2007general} and recently by Dahlgaard et. al~\cite{dahlgaard2014dynamic}.
Experimental evaluation for some labeling schemes for various properties on general graphs have been performed by Caminiti et.~al~\cite{caminiti2008engineering}, Fischer~\cite{fischer2009short} and Rotbart et.~al~\cite{rotbart2014evaluation}.

Our study also contributes to the   the graph-theory related concept of induced universal graphs. 
Given a  graph family of $n$ node graphs  $\mathcal{F}$, the graph $G$ is  induced-universal to the family if it contains all these graphs as induced sub-graphs.
The aim of this study is  to find smallest induced-universal graph possible.
Kannan, Naor and Rudich~\cite{Kannan92} showed that an $f(n) \log n$ adjacency labeling scheme for $\mathcal{F}$  constructs an induced universal graph for this family of  $2^{f(n)}$ vertices. 

\subsection{Our contribution}
For the family of power-law graphs we contribute:
\paragraph{An  $O(\sqrt[\alpha] n (\log n)^{1 - 1/\alpha})$ adjacency labeling scheme.}
The scheme is based on two ideas:
(i) a labeling \emph{strategy} that  partitions the vertices of $G$ into high (``fat'') and low degree (``thin'') vertices based on a threshold degree, and (ii) a threshold \emph{prediction} that depends only on the coefficient $\alpha$ of a power-law curve fitted to the degree distribution of $G$. 
Real-world power-law graphs rarely exceed  $~10^{10}$ vertices, implying a label size of at most  ${10^{5}}$ bits, well within the processing capabilities of current hardware. 
Our  scheme may be appealing in practice,  both due  to its simplicity and the reasonable size of its labels.
Using the same ideas, we get an  asymptotically near-tight  $O(\sqrt{n \log n})$ adjacency labeling scheme for sparse graphs.

\paragraph{A lower bound of $\Omega(\sqrt[\alpha]{n})$ for any adjacency labeling scheme.}
To this end we define a  restrictive subclass of power-law graphs and show that it is contained in the bigger class we study for the upper bound; we show that this class requires label size $\Omega(\sqrt[\alpha]{n})$ for $n$-vertex graphs.
This lower bound shows that our upper bound above is asymptotically  optimal, bar a $(\log n)^{1 - 1/\alpha}$ factor.
By the connections between adjacency labeling schemes and universal graphs, we also obtain upper and lower bounds for induced universal graphs for power-law graphs. 

\paragraph{A  $o(n)$ distance labeling scheme.}
Using similar strategy to the adjacency labeling scheme, and a small modification, we get this reult.


\paragraph{An experimental investigation  of our labeling scheme}
Using both real-world (23K-325K vertices) and synthetic (300K-1M vertices) data sets, we observe that:
(i) Our threshold \emph{prediction} performs close to optimal when using the labeling \emph{strategy} above. 
%That is true for both the maximum label size as well as for the overhead incurred by distributing  the data structure, and 
(ii) our labeling scheme achieves maximum label size several orders of magnitude smaller than the state-of-the-art labeling schemes for more general graph families.
% as well as  upper and lower bounds on the size of induced universal graphs \cite{moon1965minimal} for the class of power-law graphs. 
\vspace{\baselineskip}

%In addition, our study  may contribute to  the understanding of the quality of  \emph{generative models}---procedures that ``grow'' random graphs whose degree distributions are with high probability ``close'' to power-law graphs,  such as the Barabasi-Albert model~\cite{barabasi1999emergence} and the   Aiello-Chung-Lu model~\cite{aiello2001random}. As a first step, we provide an evidence  that the randomized Barabasi-Albert model~\cite{barabasi1999emergence} produces only a small fraction of the power-law graphs possible.








