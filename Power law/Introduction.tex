% !TEX root = WWW.tex


\section{Introduction}
A body of work on large, real-world networks  deals with the difficulties of storing them and to effectively resolve queries on them; examples of techniques are
compression~\cite{boldi2004webgraph,boldi2011layered} and dissemination  the underlying graphs of these networks over several machines~\cite{gonzalez2012powergraph, stanton2012streaming, xie2014distributed}.
Another approach is to disseminate the structural information of the graph to its vertices. This \emph{peer-to-peer} strategy allows inferring the graph's local topology using only local information stored in each vertex without using costly access to large, global data structures.
In particular, it can be useful to address privacy concerns and ensure a high survivability rate~\cite{buchegger2009peerson}.

We posit that a useful tool for such a peer-to-peer strategy is the notion of a \emph{labeling scheme}: an algorithm that assigns a bit string--a \emph{label}--to each vertex so that a query between any two vertices can be deduced solely from their respective labels. 
Labeling schemes are extremely well-studied in the algorithmic literature~\cite{alstrup2014adjacency,brady2006compact,caminiti2008engineering,dahlgaard2014dynamic,gavoillea2004distance,gavoille2007shorter,katz2004labeling,korman2007general,korman2007compact,Korman07,rotbart2014evaluation}; the main objective is to minimize the \emph{maximum label size}: the maximum number of bits used in a label of any vertex. Among applications for labeling schemes are  XML search engines~\cite{cohen2010labeling}, mapping services~\cite{abraham2011hub}, and internet routing~\cite{krioukov2004compact}.

One class of graphs extensively used for modelling real-world networks is \emph{power-law graphs}: roughly, $n$-vertex graphs where the number of vertices of degree $k$ is proportional to $n/k^{\alpha}$ for some positive $\alpha$. Power-law graphs (also called scale-free graphs in the literature) have been used to model the Internet AS-level graph \cite{DBLP:journals/ton/SiganosFFF03,DBLP:conf/podc/AkellaCKS03}, and many other types of network (see, e.g., \cite{clauset2009power,mitzenmacher2004brief} for overviews). 
The adequacy of fit of power-law graph models to actual data, as well as the empirical correctness of the conjectured mechanisms giving rise to power-law behaviour, have been subject to criticism (see, e.g., \cite{DBLP:journals/jacm/AchlioptasCKM09,clauset2009power}).
 In spite of such criticism, and because their degree distribution affords a reasonable approximation of the degree distribution of many networks, the class of power-law graphs remains a popular tool in network modelling.
In this paper, we perform the first theoretical and practical study of adjacency labeling schemes for classes of graphs whose statistical properties--in particular their \emph{degree distribution}--more closely resemble that of real-world networks.


\subsection{Our contribution}
We first define two families of graphs, one that  contains and one that is contained by the standard definitions of power-law graphs in the literature.
Using those we contribute the following results for the family of power-law graphs:

\paragraph{An  $O(\sqrt[\alpha] n (\log n)^{1 - 1/\alpha})$ adjacency labeling scheme}
The scheme is based on two ideas:
(i) a labeling \emph{strategy} that  partitions the vertices of $G$ into high (``fat'') and low degree (``thin'') vertices based on a threshold degree, and (ii) a threshold \emph{prediction} that depends only on the coefficient $\alpha$ of a power-law curve fitted to the degree distribution of $G$.  These ideas are illustrated in Figure~\ref{f:principle}.
Real-world power-law graphs rarely exceed  $~10^{10}$ vertices, implying a label size of at most  ${10^{5}}$ bits, well within the processing capabilities of current hardware. 
Our  scheme may be appealing in practice,  both due  to its simplicity and the reasonable size of its labels.
Using the same ideas, we get an  asymptotically near-tight  $O(\sqrt{n \log n})$ adjacency labeling scheme for sparse graphs.

\paragraph{A lower bound of $\Omega(\sqrt[\alpha]{n})$ for any adjacency labeling scheme}
We use our  restrictive subclass of power-law graphs and  show that it requires label size $\Omega(\sqrt[\alpha]{n})$ for $n$-vertex graphs.
This lower bound shows that our upper bound above is asymptotically  optimal, bar a $(\log n)^{1 - 1/\alpha}$ factor.
By the connections between adjacency labeling schemes and universal graphs, we also obtain upper and lower bounds for induced universal graphs for power-law graphs. 

\paragraph{An $o(n)$ distance labeling scheme}
As a theoretical contribution, we extend the ideas of our  adjacency labeling scheme to arrive to a $o(n)$ distance labeling scheme for power-law graphs.
Our labeling scheme is designed to outperform competing labeling schemes for small distances, in according to Chung and Lu's findings~\cite{chung2004average} on the small expected diameter of power-law graphs.

\paragraph{An experimental investigation  of our labeling scheme}
Using both real-world (23K-3M vertices) and synthetic (300K-1M vertices) data sets, we observe the following:
(i) Our threshold \emph{prediction} performs close to optimal when using the labeling \emph{strategy} above, in particular to graphs with  a degree distribution  close to  power-law. 
(ii) Our labeling scheme achieves maximum label size that is several orders of magnitude smaller than the state-of-the-art labeling schemes for more general graph families.


\begin{figure}
\centering
\subfloat[]{
    \includegraphics[width=0.23\textwidth]{Figures/Principle1.pdf}
}
\subfloat[]{
    \includegraphics[width=0.23\textwidth]{Figures/Principle2.pdf}
}%(
\caption{Two illustrations of the main idea: Figure (a) demonstrates the threshold assignment, figure (b) demonstrates the label assignment, in which fat (black) nodes do not store adjacency to thin (white) nodes.}
    \label{f:principle}
\end{figure}

\subsection{Related work}
Adjacency labeling schemes for  numerous important graph families are by now well understood. 
General graphs  require a label size of $n/2+O(1)$~\cite{moon1965minimal, alstrup2014adjacency}, while 
trees, planar graphs, and bounded degree graphs enjoy labels of logarithmic size~\cite{Alstrup02, gavoille2007shorter, adjiashvili2014labeling}. 

Routing labeling schemes for power-law graphs  have been investigated by Brady and Cowen~\cite{brady2006compact}, and by Chen et al.~\cite{chen2012compact}. Labeling schemes for other properties than adjacency have been investigated for various classes of graphs, e.g., distance~\cite{gavoillea2004distance}, and flow~\cite{katz2004labeling}. 
Dynamic labeling schemes were studied by Korman and Peleg~\cite{korman2007general,korman2007compact,Korman07} and recently by Dahlgaard et. al~\cite{dahlgaard2014dynamic}.
Experimental evaluations for some labeling schemes for various properties on general graphs have been performed by Caminiti et.~al~\cite{caminiti2008engineering}, Fischer~\cite{fischer2009short} and Rotbart et.~al~\cite{rotbart2014evaluation}.

In the context of distributed graph computing systems,  a somewhat related paradigm of computation is the vertex centric computing model "think like a vertex". In this model  each vertex exchanges messages  only with nearby vertices, to improve locality and simplify the design and implementation of such systems. Among the numerous systems proposed are Pregel~\cite{malewicz2010pregel}, Power-Graph~\cite{gonzalez2012powergraph} and GraphLab~\cite{low2012distributed}. For a recent survey on the topic see~\cite{mccunethinking2015}.





