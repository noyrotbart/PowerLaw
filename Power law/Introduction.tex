% !TEX root = WWW.tex


\section{Introduction}
A fundamental problem in networks is how to disseminate the structural information of the underlying graph of a network to its vertices. The purpose of such dissemination is that the local topology of the network can be inferred using only local information stored in each vertex without using costly access to large, global data structures.
One way of doing so is via  \emph{ labeling schemes}: an algorithm that assigns a bit string--a \emph{label}--to each vertex so that a query between any two vertices can be deduced solely from their respective labels. 
The main objective of  labeling schemes is to minimize the \emph{maximum label size}: the maximum number of bits used in a label of any vertex. 
Labeling schemes for adjacency and other properties have found practical use in  XML search engines~\cite{cohen2010labeling}, mapping services~\cite{abraham2011hub} and routing~\cite{krioukov2004compact}.

In this paper we are interested in particular with labeling schemes for   \emph{adjacency} queries. 
For general graphs Moon~\cite{moon1965minimal} showed lower and upper bounds of respectively $n/2$ and $n/2+\log n$ bits on the  label size.
The asymptotic gap between these bounds was only recently closed  by Alstrup et al.~\cite{alstrup2014adjacency} who proved an upper bound of $n/2 + 6$ bits. 
Upper bounds for adjacency labeling schemes exist for many specific classes of graphs, including trees~\cite{Alstrup02}, planar graphs~\cite{gavoille2007shorter},  bounded-degree graphs~\cite{adjiashvili2014labeling}, and bipartite graphs~\cite{lozin2007minimal}.

However, for classes of graphs whose statistical properties--in particular their \emph{degree distribution}--more closely resemble that of real-world networks, there has, to our knowledge, been no research on adjacency labeling schemes.
% and to our knowledge no attempts at general, space-efficient algorithms for local storage of the graph structure. 
One class of graphs extensively used for modelling real-world networks is \emph{power-law graphs}: roughly, $n$-vertex graphs where the number of vertices of degree $k$ is proportional to $n/k^{\alpha}$ for some positive $\alpha$. Power-law graphs (also called scale-free graphs in the literature) have been used, e.g., to model the Internet AS-level graph \cite{DBLP:journals/ton/SiganosFFF03,DBLP:conf/podc/AkellaCKS03}, and many other types of network (see, e.g., \cite{mitzenmacher2004brief,clauset2009power} for overviews). 
%Both the adequacy of fit of power-law graph models of networks to actual data, and the empirical correctness of the conjectured mechanisms giving rise to power-law behaviour have been subject to criticism (see, e.g., \cite{DBLP:journals/jacm/AchlioptasCKM09,clauset2009power}). 
The adequacy of fit of power-law graph models to actual data, as well as the empirical correctness of the conjectured mechanisms giving rise to power-law behaviour, have been subject to criticism (see, e.g., \cite{DBLP:journals/jacm/AchlioptasCKM09,clauset2009power}). 
In spite of such criticism, and because their degree distribution affords a reasonable approximation of the degree distribution of many networks, the class of power-law graphs remains a popular tool in network modelling whose statistical behaviour is well-understood: e.g., for power-law graphs with $2<\alpha<3$, the range most often seen in the modeling of real-world networks \cite{clauset2009power}, it is known that with high probability the average distance between any two vertices is  $O(\log \log n)$, the diameter is $O(\log n)$ and there exists a dense subgraph of $n^{c/\log \log n}$ vertices~\cite{chung2004average}. 

Routing labeling schemes for power-law graphs  have been investigated by Brady and Cowen~\cite{brady2006compact}, and by Chen et al.~\cite{chen2012compact}. Labeling schemes for other properties than adjacency have been investigated for various classes of graphs, e.g., distance~\cite{gavoillea2004distance}, and flow~\cite{katz2004labeling}. 
Dynamic labeling schemes were studied by Korman and Peleg~\cite{korman2007compact,Korman07,korman2007general} and recently by Dahlgaard et. al~\cite{dahlgaard2014dynamic}.
Experimental evaluation for some labeling schemes for various properties on general graphs have been performed by Caminiti et.~al~\cite{caminiti2008engineering}, Fischer~\cite{fischer2009short} and Rotbart et.~al~\cite{rotbart2014evaluation}.

%This paper is the first to prove upper and lower bounds for adjacency labeling schemes for power-law graphs, and one of only a few papers containing an empirical study of the efficiency of  labeling schemes on real-world graphs. 
Adjacency labeling schemes are tightly coupled with  the graph-theory related concept of induced universal graphs.
Given a  graph family  $\mathcal{F}$, the aim  is to find smallest $N$ such that a graph of  $N$ vertices contains all graphs in $\mathcal{F}$ as induced subgraphs. 
Kannan, Naor and Rudich~\cite{Kannan92} showed that an $f(n) \log n$ adjacency labeling scheme for $\mathcal{F}$  constructs an induced universal graph for this family of  $2^{f(n)}$ vertices. Some  of the adjacency labeling schemes reported earlier  contributed a better bound than was known of  induced universal graphs (see e.g~\cite{BCLR,Alstrup02}).
In the context of  sparse graphs,  a body of work on universal graphs\footnote{A graph that  contains each graph from the graph family as subgraph, not necessarily induced.} for this family was  investigated both by  Babai et al.~\cite{babai1982graphs} and  by Alon and Asodi~\cite{Alon2002universal}. 

%New
Due to their size, studying web graphs is difficult to store in main memory. This led first to compression techniques~\cite{boldi2004webgraph,boldi2011layered}, and thereafter to study on the dissemination  of such graphs over several machines~\cite{gonzalez2012powergraph}. Our study sets theoretical limits to the space required for such  efforts.

\subsection{Our contribution}

Our contributions are:

\paragraph{An  $O(\sqrt[\alpha] n (\log n)^{1 - 1/\alpha})$ adjacency labeling scheme for power-law graphs $G$}
%$\sqrt[\alpha]{C'n}(\log n)^{1 - 1/\alpha} + 2\log n + 1$ for some $C'$. 
The scheme is based on two ideas:
(I) a labeling \emph{strategy} that  partitions the vertices of $G$ into high (``fat'') and low degree (``thin'') vertices based on a threshold degree, and (II) a threshold \emph{prediction} that depends only on the coefficient $\alpha$ of a power-law curve fitted to the degree distribution of $G$. 
Real-world power-law graphs rarely exceed  $~10^{10}$ vertices, implying a label size of at most  ${10^{5}}$ bits, well within the processing capabilities of current hardware. 
We claim that our  scheme is thus appealing in practice   due both to  its simplicity and hte small size of its labels.
Using the same ideas, we get an  asymptotically near-tight  $O(\sqrt{n \log n})$ adjacency labeling scheme for sparse graphs.
% and five orders of magnitude less than the optimal upper bound of $.5 \cdot 10^9$ bits for general graphs \cite{alstrup2014adjacency}. 

\paragraph{A lower bound of $\Omega(\sqrt[\alpha]{n})$ bits on the maximum label size for any adjacency labeling scheme for power-law graphs}
To this end we define a  restrictive subclass of power-law graphs and show that it is contained in the bigger class we study for the upper bound; we show that this class requires label size $\Omega(\sqrt[\alpha]{n})$ for $n$-vertex graphs.
% use  the Erd{\"o}s-Gallai Theorem \cite{Erdos1960graphs} and construct  a highly restrictive subclass of power-law graphs.
This lower bound shows that our upper bound above is asymptotically  optimal, bar a $(\log n)^{1 - 1/\alpha}$ factor.
By the connections between adjacency labeling schemes and universal graphs, we also obtain upper and lower bounds for induced universal graphs for power-law graphs. 


\paragraph{An experimental investigation  of our labeling scheme}
Using both real-world (23K-325K vertices) and synthetic (300K-1M vertices) data sets, we observe that:
(i) Our threshold \emph{prediction} performs close to optimal when using the labeling \emph{strategy} above. 
%That is true for both the maximum label size as well as for the overhead incurred by distributing  the data structure, and 
(ii) our labeling scheme achieves maximum label size several orders of magnitude smaller than the state-of-the-art labeling schemes for more general graph families.
% as well as  upper and lower bounds on the size of induced universal graphs \cite{moon1965minimal} for the class of power-law graphs. 
\vspace{\baselineskip}

In addition, our study  may contribute to  the understanding of the quality of  \emph{generative models}---procedures that ``grow'' random graphs whose degree distributions are with high probability ``close'' to power-law graphs,  such as the Barabasi-Albert model~\cite{barabasi1999emergence} and the   Aiello-Chung-Lu model~\cite{aiello2001random}. As a first step, we provide an evidence  that the randomized Barabasi-Albert model~\cite{barabasi1999emergence} produces only a small fraction of the power-law graphs possible.








